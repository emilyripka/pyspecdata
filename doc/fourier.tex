% this is the original source (.rst made from this)
\begin{enumerate}
    \item Set the \texttt{FT} property appropriately, and changing the units of the axis appropriately.
    \item Identify whether the $u$ or $v$ domain is the original ``source'' of the signal, and which is derived from the source.
        By default assume that the source is not aliased.
        In this way, I can automatically marked whether an axis is
            assumed to be ``safe'' (i.e. ``not aliased'') or not.
        This is relevant when performing time- (frequency-)shifts that
            are not integral multiples of $\Delta u$ ($\Delta v$).
    \item Pull the $u$-axis and determine $\Delta u$.
    \item Calculate (only) the initial $v$-axis (starting at 0),
        the post-shift needed to get the $v$-axis I want (based on \texttt{FT_start_v}),
        any full-SW aliasing that's needed to get the $v$-axis that I want,
        and the post-shift discrepancy (not accounted for by the combination of
        the integral post-shift and the full-SW aliasing) needed to get the
        $v$-axis I want (this becomes a $u$-dependent linear phase shift)
    \item If I do a traditional shift (\textit{i.e.}, like \texttt{fftshift}), mark as \texttt{FT_v_not_aliased}.
    \item If there is a post-shift discrepancy, check that the $u$-axis is
        balanced, then apply the post-shift discrepancy as a linear phase shift
        along $u$.
    \item Zero-fill before any pre-shifting, since zeros should be placed at large positive frequencies.
    \item Pre-shift to place the origin at the beginning of the axis, aliasing
        the negative frequencies to the right of the largest positive
        frequencies, and store any discrepancy that's not accounted for by a
        the shift (which is an integral multiple of $\Delta u$).
    \item Perform the FFT and replace the axis with the initial $v$.
    \item Since the data has already been $v$-shifted using the linear phase
        shift along $u$ above, change the $v$-axis to reflect this.
    \item Adjust the normalization of the data (this depends on whether we are doing \texttt{ft} or \texttt{ift}).
        \begin{itemize}
            \item As of now, the ft$\Rightarrow$ift is not invertible, because I have defined them as an integral over the exponential only; I might later consider dividing the ft by the record length ($T$) to return the original units.
        \end{itemize}
    \item If there was any pre-shift discrepancy, apply it as a phase-shift along the $v$-axis.
\end{enumerate}
