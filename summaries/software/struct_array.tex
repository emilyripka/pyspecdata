\paragraph{textlabel\_bargraph(mystructarray)}
This plots data as a bargraph.
It uses all fields with text-format labels to form nested groups of the data.

For example, we can load the following text into test.csv.

\begin{verbatim}
chemical,run,measurement A,measurement B
A,1,10,20
A,2,11,20.5
A,3,10.5,30
B,1,20,5.1
C,1,30,1.0
C,2,30.5,1.1
\end{verbatim}

Then read the data with the matplotlib function csv2rec,
    and plot it.
Here, we use some functions that will be introduced later in order
    to convert a set of numbered labels, initially simply entered as numbers,
    to text, so that the code will also use those as a sorting dimension.

<<<<<<< HEAD
\begin{mykwargs}
    \begin{description}
        \item[othersort = None] fields that might not be text format (and so not used for the labels), but that you want to treat as such
        \item[spacing = 0.1]  the spacing between groups of bars
        \item[ax = None] an existing axis to plot on
        \item[tickfontsize = 8] the size of the tick labels
    \end{description}
\end{mykwargs}

=======
>>>>>>> public
\begin{python}
#redo!!!!!!!!!!!!!!!!!!!
from matplotlib.mlab import rec2csv, csv2rec
data = csv2rec('test.csv')
lrecordarray(data)
# the next two lines are just to convert the ``run'' number to a text field
data = lambda_rec(data,'run_string',lambda x: '%0.1f'%x,'run') # make a text field
data = data[list(set(data.dtype.names)-set(['run']))] # and pop the old run number field
lrecordarray(data)
textlabel_bargraph(data,verbose = True)
lplot('bargraph_demo120921.pdf')
\end{python}

\begin{python}
import matplotlib.patches as mpatches
import matplotlib.pyplot as plt

styles = mpatches.ArrowStyle.get_styles()

ncol=2
nrow = (len(styles)+1) // ncol
figheight = (nrow+0.5)
fig1 = figure(1, (4.*ncol/1.5, figheight/1.5))
fontsize = 0.2 * 70

ax = fig1.add_axes([0, 0, 1, 1], frameon=False, aspect=1.)

ax.set_xlim(0, 4*ncol)
ax.set_ylim(0, figheight)

def to_texstring(s):
    s = s.replace("<", r"$<$")
    s = s.replace(">", r"$>$")
    s = s.replace("|", r"$|$")
    return s

for i, (stylename, styleclass) in enumerate(sorted(styles.items())):
    x = 3.2 + (i//nrow)*4
    y = (figheight - 0.7 - i%nrow) # /figheight
    p = mpatches.Circle((x, y), 0.4, fc="w")
    ax.add_patch(p)

    ax.annotate(to_texstring(stylename), (x, y),
                (x-1.5, y),
                #xycoords="figure fraction", textcoords="figure fraction",
                ha="right", va="center",
                size=fontsize,
                arrowprops=dict(arrowstyle=stylename,
                                patchB=p,
                                shrinkA=5,
                                shrinkB=5,
                                fc="w", ec="k",
                                connectionstyle="arc3,rad=0.0",
                                ),
                bbox=dict(boxstyle="square", fc="w"))

ax.xaxis.set_visible(False)
ax.yaxis.set_visible(False)

lplot('test_arrows.pdf')
\end{python}
\paragraph{applyto\_rec(myfunc,myarray,['name1',\ldots,'nameN'])}\label{codelabel:applyto_rec}
This applies \texttt{myarray} to \texttt{myarray} in an attempt
    to collapse many datapoints describing the same set of data
    into a single set of datapoints.

Sometimes, you have a structured array \texttt{myarray} that
    contains large sets of data,
    and you are only interested in the average
    (\texttt{myfunc} is \texttt{mean})
    or in the standard deviation
    (\texttt{myfunc} is \texttt{std})
    of that data.
For such cases, this function allows you to identify ``unique''
    pieces of data according to one or more fields
    (the \texttt{name1}\ldots\texttt{nameN} above),
    and to find the mean, standard deviation, etc,
    over those means.

For instance:
\ntd{make function, and run test function}

\begin{python}
obs("just make a fake set of data, which I'm going to identify by ``xvalue'' and ``yvalue''")
b = r_[1.0,2.0,3.0,4.0,
    1.0,2.0,3.1,4.1,
    1.0,2.0,3.2,4.2,
    2.0,2.0,2.0,4.0,
    2.0,2.0,2.2,4.2,
    2.0,3.0,2.0,4.0,
    2.0,3.0,3.2,5.2]
a = make_rec(b[0:4].tolist(),['xvalue','yvalue','datapoint1','datapoint2']) # just to set the type
b = b.view(a.dtype)
lrecordarray(b)
b_mean = applyto_rec(mean,b,['xvalue','yvalue'])
b_std = applyto_rec(std,b,['xvalue','yvalue'])
obs("after the mean:")
lrecordarray(b_mean)
obs("after the std:")
lrecordarray(b_std)
\end{python}

\paragraph{meanstd\_rec(myfunc,['name1',\ldots,'nameN']}
\nts{For the example code, run a search over the inprocess/crowding tex files}
\paragraph{make\_rec(\ldots)}
<<<<<<< HEAD
This makes a numpy structured array.
Originally, it was designed to generate only 1-element array.
Now, however, if it's fed matching N-dimensional inputs,
    it will assume that they form an N-dimensional array of outputs.

Given a couple example variables, it can also be used to construct a new structured array, for later data loading (see kwargs below).

=======
This makes a 1 element numpy structured array.
>>>>>>> public
It can be called in one of two ways.

\subparagraph{make\_rec(inputdict)} where ``inputdict'' is a dictionary
    whose key,value pairs give the field names and their values.

\subparagraph{make\_rec(inputlist,namelist)}
    where ``inputlist'' g
\begin{mykwargs}
    \begin{description}
        \item[strlen = 100] gives the maximum length of strings
            (in structured arrays, the maximum length of strings needs to be
            specified)
        \item[order = ['field1','field2',\ldots] ]
            specifies the order in which the fields are stored,
            where ``fieldN'' here is the name of a field.
            If all the fields are not specified, the specified fields
            will come first in the order, and the remainder will come afterwards.
<<<<<<< HEAD
        \item[zeros_like = False] if this is anything other than ``False,'' construct a new structured array of shape \texttt{zeros\_like} 
        \item[equal\_shapes] ???? 
=======
>>>>>>> public
    \end{description}
\end{mykwargs}
If a structured array with more elements is desired,
    it can be constructed by using the data type, or assigning values of the 1-D array,
    \nts{like this\ldots}
\paragraph{lookup\_rec(\ldots)}
\nts{This is made obsolete by the decorate\_rec function!}
\paragraph{reorder\_rec(myarray,['name1',\ldots,'nameN'])}
Reorder the fields in the structured array named myarray,
    placing the fields named 'name1',\ldots,'name2'
    first and in order.
<<<<<<< HEAD
\paragraph{rename\_fields(myarray,{`oldname':`newname'})}
This is from numpy.lib.recfunctions, and it renames the fields.
=======
>>>>>>> public
\paragraph{lambda\_rec(myarray,\ldots)}\label{codelabel:lambda_rec}
This function is used to perform spreadsheet-like calculations,
    where we evaluate functions of the fields in the structured
    array named myarray and use the result to generate new fields.

\subparagraph{lambda_rec(ksp_uncorr_data,'s_{max}',calc\_s,'concentration')}
In this example, the function calc\_s takes a single argument, which corresponds to the concentration field.
\subparagraph{lambda\_rec(myarray,``newfieldname'',(lambda x: sin(x)+1.0),``inputfield'')}
In this example, we have used a ``lambda'' function;
    lambda functions are a standard part of the python language.
This takes the structured array myarray,
    and evaluate $\sin(x)+1.0$, where $x$ is given
    by the data in the field named ``inputfield.''
Place the result in a new field named ``newfieldname,''
    which is placed after the last argument.
If there is already a field called ``newfiledname,''
    that field will be replaced with the result.

\subparagraph{lambda\_rec(myarray,``newfieldname'',(lambda x: sin(x)+1.0))}
Same as above, but where \texttt{``inputfield''} is the same as \texttt{``newfieldname''}.
\subparagraph{lambda\_rec(myarray,``newfieldname'',(lambda x,y: sin(x)+cos(y)+1.0),[``inputfield1'',``inputfield2''])}
This does the same as the previous, except that
    the result is $\sin(x)+\cos(y)+1.0$, where
    $x$ is given by inputfield1 and $y$ by inputfield2.
An arbitrary number of arguments can be used in this way.
\paragraph{decorate\_rec(\ldots)}
Typically, we want to store information in an HDF5 file consistently
    and efficiently.
In the examples here,
    we might want to store a bunch of information about a chemical
    in one table, for instance, the chemical name, the spin label
    concentration, etc.
We can then retrieve that table as a structured array
    with the PyTables .read() method
    (or retrieve information that matches a pattern with .readWhere(pattern)).
The read function returns a structured array,
    in these examples, we assume that we we have assigned it
    to the variable chemical\_table.
Then, in other tables,
    we can simply refer, eg., to the ``index'' field of, eg., the chemical\_table 
    in order to refer to all the relevant information at once.

The ``decorate\_rec'' allows us to join all the information back onto the table
    that uses the index.

\subparagraph{decorate\_rec((tableA,``chemical\_index''),(tableB,``index'')}
join all the fields from tableB, whose index field is called ``index'',
    onto the tableA.
If the names of any of the fields overlap, do not copy the information from
    tableB.
This is somewhat analogous to an SQL join.

\subparagraph{decorate\_rec((t1\_data,[``run\_number'',``chemical'']),(emax\_data,[``run\_number'',``chemical''])}
Here we take a table of $T_1$ data, and join the $E_{max}$ data for the same run\_number
    and chemical name onto it.
This form (where an arbitrary number of fields can be used in the list above)
    is useful when there is no unique index that can be used to
    collect together matching data from different tables
    when there is not a unique index field that identifies
    the row by experiment/chemical/etc.
\subparagraph{keyword arguments:}
\begin{description}
    \item[drop\_rows = False] if there is no information in the second argument to match that in the first, an error is thrown.  Rather than ``False'' this can also be set to:
        \begin{description}
            \item[True] the rows are dropped, and warning messages are issued.
            \item['return'] the dropped rows are returned as a second argument
        \end{description}
\end{description}

For example
\begin{python}
# rerun!!
a = r_[make_rec([1,2],['key','first']),
    make_rec([2,3],['key','first'])]
print "let's use a first table like this:"
lrecordarray(a)
b = r_[make_rec([1,4,'something'],['key','first','second']),
    make_rec([2,3,'something else'],['key','first','second']),
    make_rec([2,3,'something else again'],['key','first','second'])]
print 'and a second argument like this:'
lrecordarray(b)
print "now, let's decorate the first with the second, using the ``key'' column to identify a unique piece of data:"
lrecordarray(decorate_rec((a,'key'),(b,'key')))
obs("Notice how it keeps the data from the first table when there's overlap\n\n")
obs("Also notice how if there is more than one row in the second table that matches a row in the first, it makes a duplicate.")
print "now, let's be more specific, and insist that both ``key'' and ``first'' must match:"
lrecordarray(decorate_rec((a,['key','first']),(b,['key','first']),
    drop_rows = True))
obs('Note that this would refuse to run and throw and error without setting drop\_rows to True!')
print "Finally, let's go ahead and pull out the dropped rows, for further processing" 
good,bad = decorate_rec((a,['key','first']),(b,['key','first']),
    drop_rows = 'return')
obs('The stuff that matched is:')
lrecordarray(good)
obs("The stuff that didn't find a match is:")
lrecordarray(bad)
\end{python}
\paragraph{newcol\_rec(myarray,\ldots)}
Adds new, empty fields according to the dtype specifier,
    which is given by the second argument.
The data in the new fields is a numpy ``empty'' array
    before it's assigned;
    it is unassigned memory,
    i.e. jibberish.

\subparagraph{newcol\_rec(myarray,[('first','<f8'),('second','i1'),('third','i1',(3,3))])}
This is the format that we recommend.
This example adds three fields name ``first,'' ``second,'' and ``third.'' 
The datatype of the first field is an eight-byte float (f8) and explicitly
    little-endian ($<$).
The second field is a single-byte long integer.
The third field is a 3x3 array of integers.

\subparagraph{newcol\_rec(myarray,someothervar.dtype)}
This will use the dtype of an existing variable.

\subparagraph{newcol\_rec(myarray,\ldots)}
There are various ways of specifying a dtype, and any of them is acceptable
    for the second argument.
They are further explained in the numpy documentation.

\begin{python}
obs('for example:')
A = make_rec([1.0,200,'something'],['first','second','third'])
A_row2 = A.copy()
A_row2['first'] = 10.0
A_row2['second'] = 1
A_row2['third'] = 'another thing'
A = r_[A,A_row2]
obs('take this array:')
lrecordarray(A)
obs('and add a column called ``fourth\'\'')
B = newcol_rec(A,('fourth','<f8'))
lrecordarray(B)
obs('or two columns')
B = newcol_rec(A,[('fourth','<f8'),('fifth','<i8')])
lrecordarray(B)
obs('note how the extra column is just filled with random junk')
\end{python}

\paragraph{{\scriptsize in module fornotebook:}lrecordarray(myarray)}
This prints a spreadsheet array
    as a latex tabular environment (in spreadsheet-like format).
It takes the optional keyword arguments
\begin{mykwargs}
    \begin{description}
        \item[columnformat = True] prints the fields as columns, rather than as rows.
        \item[smoosh = True] \nts{I don't know what this does $\Rightarrow$ something with printing in the given space?}
        \item[multi = True] merge cells in adjacent rows that have the same value.
    \end{description}
\end{mykwargs}
\paragraph{{\scriptsize in module fornotebook:}lrecordarray\_broken(myarray)}
