<<<<<<< HEAD
\subsection{Biol. Demo Protocol}\label{sec:biol_}
\timeblockstart
\timeblocktotal{3.6}
Here, I am demonstrating biological samples at 10\uM.

\subsubsection{initial setup}\maxminutes{57}
\paragraph{first time}
=======
\subsection{tau187 $T_{1,0}$ sample}\label{sec:biol_tau_unlabeled}
\timeblockstart
\timeblocktotal{3.6}
This is the protocol

\subsubsection{initial setup}\maxminutes{57}
\paragraph{before leaving}
>>>>>>> public
Be sure to carry over:
\begin{itemize}
    \item microcentrifuge tubes
    \item toolbox 
    \item multimeter 
    \item samples (leave in my box in CNSI fridge)
    \item magnifying glasses
\end{itemize}

<<<<<<< HEAD
$B_0$ field off, air off, everything else on, and swap the dewar for the quartz tube.

Attach air tube to temperature control, insert, turn on temp control.

=======
>>>>>>> public
\paragraph{always}
Be sure to change section title and label!!!

Copy section label to list for the day.

<<<<<<< HEAD
$B_0$ field off, air off.

\paragraph{capillary prep}
\subparagraph{using the sealed tubes}

Cut a decent length of tube.

Put capillary on GELoader tip, and turn pipette upside down.

Centrifuge sample down at 1000 rpm.

Cut again, then add critoseal and withdraw air with pipette

Put stuff back in fridge!

Measure sample.

Insert in probe.

\subsubsection{Quick ESR in probe}\maxminutes{11}
Put into tune mode, $40\;dB$ first.

Insert probe, and turn up air to 12~SCFH (since a new sample, just use a higher flow rate).

Move to near 9.77 and autotune.

\paragraph{test for signal}
try same parameters as her.

try to increase the modulation amplitude, just to see signal.

zoom in and repeat.

check for saturation 15, down by 3~dB.

save saturated scan.
\fn{dna_cs14_overmod_sat_120314}

use a power slightly before saturation.

Make an experiment with this power, but turn down the modulation amplitude and change the receiver window and the resolution and number of scans.

Increase the modulation amplitude to 2~G.

=======
\paragraph{On + set up for (first exp only)}
\precaution{Check that copper plate is set up}

EPR and source on, magnet and air off.

Be sure mod coil is hooked up, and the waveguide switch is set to ESR.

Remove top collets on dewar.

Take box around back.

Remove dewar.

Insert air tube (put only bottom collet on, tighten very loosely, then add and tighten top collet).

Zip tie air tube.

Attach tuning box to plate.

Wait until bridge stops flashing, then open WinEPR + switch to tune mode 40~dB.

Find dip near 9.8~GHz.

\paragraph{load sample}
Put spectrometer into tune mode, and turn down power.

Load 3.5\uL of sample.

Press twice in critoseal.

Score other end very lightly.

Break + press twice in semi-soft (clay consistency) wax.

Melt wax.

Clean w/ razor.

Inspect + measure\sout{ with magnifying glasses}.

Put into tune mode before inserting sample, and turn down power.

Insert probe, and turn up air to \sout{12~SCFH}14~SCFH (which is what I have been using) (since a new sample, just use a higher flow rate).

Move to near 9.77 and autotune.

\subsubsection{Quick ESR in probe}\maxminutes{11}
\paragraph{test for signal}
Open an old experiment.

Copy.

Try to increase the modulation amplitude, just to see signal.

zoom in and repeat.

Check for saturation starting from 15, down by 3~dB.

Save saturated scan.
\fn{tau_overmod_sat_120329}

Use a power slightly before saturation.

Make an experiment with this power, but turn down the modulation amplitude and change the receiver window and the resolution and number of scans.

>>>>>>> public
\paragraph{actual ESR}
Turn on field.

Copy experiment.
<<<<<<< HEAD
\fn{dna_cs14_overmod_120314}
=======
\fn{tau_120329.par}
>>>>>>> public

\paragraph{always}
Run actual scan.

Save ESR.
<<<<<<< HEAD
\fn{dna_cs14_bound_overmod_120314}
=======
\fn{tau_120329}
>>>>>>> public

Cntrl-S after scan is finished.

Hit Cntrl-A for ssh transfer and process.

<<<<<<< HEAD
=======

>>>>>>> public
\begin{tiny}
\begin{lstlisting}
fl = figlistl()
standard_epr(dir = DATADIR+'cnsi_data',
<<<<<<< HEAD
    files = ['dna_cs14_overmod_120314',
        'dna_cs14_bound_overmod_120314',
        'dna_cs14_nl_bound_overmod_120314',
        'dna_cs14_unbound_overmod_120315'],
=======
    files = ['tau_120329'],
    background = 'tau_nl_120329',
>>>>>>> public
    figure_list = fl)
fl.show(thisjobname()+'.pdf')
\end{lstlisting}
\end{tiny}

<<<<<<< HEAD
\paragraph{determine ratio}
\ntd{I don't have git set up right here, but later, I should paste the ratio procedure here}
\subsubsection{DNP Experimental Setup}

\paragraph{tune NMR, when I'm not using microwave power}
Set switch to DNP amp, set bridge to standby.

Attach probe w/ coil perpendicular to $B_0$ + at correct height, then turn on air (14~SCFM).

Turn on $B_0$ field.

Make a new NMR experiment based on template, and change into that experiment.

(leaving sfo1 at the value you used for DNP) \texttt{wobb}. \adlin{back moves lf/right front moves down/up}

\texttt{p1 1u}, then use \texttt{jf\_setmw} to center resonance in \texttt{gs}.

\paragraph{tune and set resonant field}
Attach probe w/ coil perpendicular to $B_0$ + at correct height, then turn on air (20~SCFM).

Put into tune mode before inserting sample, and turn down power.

Move to near 9.77 and autotune + turn on $B_0$ field.

Make a new NMR experiment based on template, and change into that experiment.

\fn{dopc_984mM_120503}

Set sfo1 to (ppt value)*(YIG frequency), then \texttt{wobb}. \adlin{back moves lf/right front moves down/up}

Check the ESR tune by hand up to 0~dB and record microwave frequency.
\begin{python}[off]
calcfielddata(9.790094,'mtsl','cnsi')
\end{python}
Put ESR bridge in standby mode and disconnect the mod coil.

Set the field to a reasonable value (last value usually fine, if not use the value from the ratio).

Use \texttt{jf\_setmw} to set frequency ratio and YIG frequency.
Check that the YIG's DC power supply reads 0.26-0.27~A with first light on and other lights green.

\texttt{jf\_zg} for signal about 15 high, then set it on resonance (lightning bolt o1).

Use \texttt{jf\_setmw} to center resonance in \texttt{gs}.

\paragraph{calibrate 90 time and $T_1$}

Run \texttt{jf_zg}; zoom in; \texttt{dpl1}; then \texttt{paropt} with \texttt{p1},8,1,3 which gives signal at 8\us, 9\us, and 10\us pulse length, which should be $\approx 360^o$.

Flip the waveguide switch so the amp is connected (not the bridge).

If needed, run a rough experiment to determine the $T_1$ time, and put it in experiment 101.

\begin{scriptsize}
\begin{python}[off]
# these change
name = 'dopc_32mMnl_120504' # the name of the experiment directory
path = DATADIR+'franck_cnsi/nmr/' # the name of the directory where you are storing your data
=======
\subsubsection{set up DNP}\maxminutes{38}
\paragraph{determine ratio (if desired)}

\subparagraph{determine experimentally}
Re-run ESR with probe.

Copy template.
\fn{tau_120329}

\sout{ Wobb + just set sfo1 to bottom of dip. }
sfo1 to 1.51671*frequency, regardless of exact nucleus.
wobb + tune NMR.

Tune EPR.

Unclick g=2, and set mod amp high so I can get signal in a signal shot.

Re-run EPR zoomed in to center peak.

Turn modulation amp down to 0.3~G, and let it run a few averages.

Set offset based on high and low peaks.
\fn{tau_findfield_120329}

Set field to zero-crossing of center peak on EPR.

Set EPR to standby.

jf\_zg.

Unplug mod coil.

Set sfo1 to resonance.

Increase aq to 2.0.  

jf\_zg and lightning bolt for o1.  

jf\_zg to check.

Copy microwave frequency from ESR and sfo1 from center frequency.  


\begin{lstlisting}
microwave_frequency = 9.789871 #(from file)
field = 3488.66
nmr_frequency = 14.8497568
field = field * 1e-4
chemical = 'mtsl'
#obs('previous:')
#calcfielddata(microwave_frequency,chemical,'cnsi')
obs('new:')
save_data({chemical+'_elratiocnsi':microwave_frequency/field})
save_data({chemical+'_nmrelratio':nmr_frequency/microwave_frequency})
calcfielddata(microwave_frequency,chemical,'cnsi')
\end{lstlisting}


Set \texttt{aq} back to 0.5.

\paragraph{Tune and set resonant field}
Detach mod coil and hang in the plastic loop (here so that I don't potentially mess up microwave tune later!).

Copy nmr template dataset.
\fn{dna_cs14_bound_120314}

Overwrite exp 1 with desired parameters.
\fn{dna_cs14_rerun_120314}

Round nearest 0.0005\GHz to get YIG frequency.

Open first NMR experiment in Topspin.

\precaution{If a different frequency or label than usual, set NMR sfo1 to (ratio)*(YIG frequency).}
Tune NMR with {\tt wobb} (type {\tt stop} when done).

\precaution{Run jf\_dnpconf if necessary -- specifically may need to decrease my attenuation settings by 2.7 $dB$ (or just round it to 3 $dB$).}

Tune microwave by hand
{\small (Simultaneously set:
\begin{itemize}
    \item bias (diode centered at 50 $dB$)
    \item frequency (AFC)
    \item phase (coarse: dip in tune mode; fine: furthest right diode w.r.t.signal phase while AFC remains locked)
    \item iris (diode centered at all powers) 
\end{itemize}
up to 5 $dB$.)}
\begin{lstlisting}
calcfielddata(9.789519,'mtsl','cnsi')
\end{lstlisting}

Check that YIG frequency is still good

EPR to standby.

\paragraph{ Match NMR + YIG frequencies}
\precaution{If frequency is really different for some reason, I will have to set the field here to the value given above.}
Set static field.

jf\_zg $\Rightarrow$ for 3.5 $\mu L$ sample, peak should 10-15 high.

Check that the source is on, and reads 0.26 $A$ (rather than 0.54 $A$).

Click lightning bolt to set o1 (while zoomed in with top window).

Set YIG frequency with \texttt{jf\_setmw}.

Be sure the ppt value is set to the value above.

Open \texttt{gs}, click the checkbox on top, and set phase correction mode do pk. 

\precaution{If gs shows an FID, then click the button on top to get FT.}
\precaution{Test amp switch with jf\_setmw (leave other parameters as default).}
Iteratively adjust the field.
{\small
\begin{enumerate}
    \item slide slider so peak is on average at 0. 
    \item click save. 
	\item jf\_setmw (31.5 $dB$, YIG frequency),
	\item then if field offset is significant, adjust $B_0$, otherwise stop
	\item jf\_zg
\end{enumerate}
}

Since the gs procedure is new,
run jf\_zg to check that the peak is roughly centered,
and also re-run jf\_setmw to check that it
reads the same number as the last go above.

\paragraph{90 time and ready amp}
{\tt jf\_zg} + zoom + {\tt dpl1} + run {\tt paropt} (p1,8,1,3)

\precaution{This should go down to up at the full cycle;
if problems, (p1,1,1,11).}
\precaution{If you want to decrease the narrow noise spikes, minimize WinEPR + turn off bridge + put all bridge cables on top of dewar while paropt is running; could be obsolete with box?}
Flip waveguide switch.

Set + record $t_{90}$ (``\texttt{p1}'')=(length of full cycle [\us])/4 (At this point, you've determined the resonance frequency, resonant field, and $t_{90}$).

\paragraph{Estimate $T_1$ (with heating)}

\subparagraph{Estimate based on concentration {\tiny (if $T_1$ is available for a different concentration)}}

\begin{lstlisting}
# edit following
oldconc = 2e-3 # set this to None if oldt1 is given at
#the same concentration
oldt1 = 0.89
newconc = 10e-6
# leave the following alone
def estimate_t1(concentration,value,newconc):
    water = 1./2.6
    relaxivity = 1./value
    relaxivity -= water
    relaxivity /= concentration
    relaxivity *= newconc
    relaxivity += water
    return 1./relaxivity
def estimate_hot_t1(thist1):
    water = 1./2.6
    hot_water = 1./3.65 # this is for the
    #new ``closed'' type probe # 7/8 adjusted this
    thist1 = 1./thist1 # convert to a rate
    thist1 -= water # figure out which part
    #is from water
    thist1 += hot_water # add back in for heating
    return 1./thist1
if oldconc == None:
    t1 = oldt1
else:
    t1 = estimate_t1(oldconc,oldt1,newconc)
    obs(dp(t1,2),r' $s$ without heating')
obs(dp(estimate_hot_t1(t1),2),r' $s$ with heating')
\end{lstlisting}

\subparagraph{Estimate experimentally (from rough measurement of $T_1$ and known heating)}

\texttt{re 1 1}

\texttt{jf\_t10} with default parameters (should be min 1,max 3,steps 5,min ratio 0.08 (should have been 0.05 to match), max ratio 4, pull from exp 4, put in exp 101).

Copy the nmr data.

Change file name!

Process the $T_1$


\begin{scriptsize}
\begin{lstlisting}
# these change
name = 'tau_120329'
path = DATADIR+'cnsi_data/'
>>>>>>> public
# the following stays the same
dnp_for_rho(path,name,[],expno=[],t1expnos = [101],
        integration_width = 150,peak_within = 500,
        show_t1_raw = True,phnum = [4],
        phchannel = [-1],
        h5file='t1_estimation_only.h5',
        pdfstring = name,
        clear_nodes=True)
t1 = retrieve_T1series('t1_estimation_only.h5',name)
def estimate_hot_t1(thist1):
    water = 1./2.6
<<<<<<< HEAD
    hot_water = 1./4.2 # this is for the new ``closed''
=======
    hot_water = 1./3.8 # this is for the new ``closed''
>>>>>>> public
    #type probe # 7/8 adjustted this
    thist1 = 1./thist1 # convert to a rate
    thist1 -= water # figure out which part is from
    #water
    thist1 += hot_water # add back in for heating
    return 1./thist1
obs(r'Min $T_1\approx$',lsafe(t1['power',0]),r'\quad $T_{1,max}\approx$',
        lsafe(estimate_hot_t1(t1)),
        r' $s$ with heating')
<<<<<<< HEAD
\end{python}
\end{scriptsize}
\paragraph{determine $T_1(t)$}
Use jf\_dnp to determine number of points for a scan 17~min long:
DOPC 32\mM $\Rightarrow$ min 3.4 of max of 3.6

Run jf\_t10s:
Set 15 experiments at 17~min for 2h50m, exp 4$\Rightarrow$ 501.
=======
\end{lstlisting}
\end{scriptsize}

\paragraph{determine $T_1(t)$}
Use jf\_dnp to determine number of points for a scan \sout{17~min}12~min long:
tau187 nl $\Rightarrow$ min of 2.1 max of 2.2

Run jf\_t10s:
Set \sout{ 20 }10 experiments for \sout{ 4:40 }\add{2:50}, exp 4$\Rightarrow$ 501.
>>>>>>> public

Come back after it ran.

\paragraph{run DNP}
<<<<<<< HEAD
Start jf\_dnp, and write down $T_1$ times entered.
bound CS19 $\Rightarrow$ 1.5 to 2.1 to be safe
DOPC high concentration $\Rightarrow$ 0.67 to 0.8 to be safe

\subsubsection{Processing}

\paragraph{process DNP}
Process results.


\begin{scriptsize}
\begin{python}[off]
import textwrap # don't pay attention to this
# change the following parameters
name = 'enterexperiment' # replace with the name of the experiment
chemical_name = 'enterchemicalhere' # this is the name of the chemical i.e. 'hydroxytempo' or 'DOPC', etc.
run_number = 120503 # this is a continuous run of data
concentration = 984e-6 # the concentration of spin label
dontfit = False # only set to true where you don't expect enhancement
path = DATADIR+'franck_cnsi/nmr/'
# if after setting all these correctly,
# it complains about your T1 experiments
# try to uncomment the t1mask line below
###########################
# the following stays the same
#search_delete_datanode('dnp.h5',name)
=======
Start jf\_dnp: Set minimum and maximum $T_1$ values based on $T_1$ estimate (with heating).
tau187 $\Rightarrow$ min 1.8, max 2.4
tau187 nl $\Rightarrow$ min of 2.1 max of 3.0

Set number of $T_1$ time to \sout{17~min}12~min for this sample.

Set watch timer for experiment time.

\nts{consider closing and opening FTDI connection in the code, so the lock check works!}
\subsubsection{wait for DNP to run}\maxminutes{90}

\subsubsection{process + put back system}\maxminutes{6}
\paragraph{process DNP}
Transfer NMR files.

Change name, chemical names, concentration, and run number.

For $T_{1,0}$, set dontfit to True, otherwise False.

\nts{why does the $T_{1,0}$ have a concentration, since it will always be 0?}
Run processing.

Mask/unmask $T_1$'s as necessary.

\nts{in the new version, I should really set it up so that it automatically takes care of both the auto steps for $T_1$ and the $T_1$ mask!, and also checks that the max $T_1$ is OK!!!}
Check that my longest $T_1$ falls within range.

Add to the compilation, to see how the data looks.

Check the consistency of the enhancements with decreasing power.

Redo experiment 5 as 505 \texttt{re 5 1;Ctrl-N} + enter exp 505 +\texttt{zg}.

Comment {\tt search\_delete\_datanode}.

\nts{should actually be able to get rid of the guessing problem by switching it to guess with the pseudoinverse!}

\begin{scriptsize}
\begin{lstlisting}
import textwrap
name = 'tau_120330'
chemical_name = 'tau187'
run_number = 120330
concentration = 600e-6
dontfit = False
# the following stays the same
path = DATADIR+'franck_cnsi/nmr'
search_delete_datanode('dnp.h5',name)
>>>>>>> public
# leave the rest of the code relatively consistent
#{{{ generate the powers for the T1 series
print 'First, check the $T_1$ powers:\n\n'
fl = []
t1_dbm,fl = auto_steps(path+name+'/t1_powers.mat',
    threshold = -35,t_minlength = 5.0*60,
    t_maxlen = 40*60, t_start = 4.9*60.,
    t_stop = inf,first_figure = fl)
print r't1\_dbm is:',lsafen(t1_dbm)
lplotfigures(fl,'t1series_'+name)
print '\n\n'
t1mask = bool8(ones(len(t1_dbm)))
# the next line will turn off select (noisy T1
# outputs) enter the number of the scan to remove --
# don't include power off
<<<<<<< HEAD
#t1mask[-1] = 0 # this is the line you sometimes want to uncomment
=======
t1mask[-1] = 0
>>>>>>> public
#}}}
dnp_for_rho(path,name,integration_width = 160,
        peak_within = 500, show_t1_raw = True,
        phnum = [4],phchannel = [-1],
        t1_autovals = r_[2:2+len(t1_dbm)][t1mask],
        t1_powers = r_[t1_dbm[t1mask],-999.],
        power_file = name+'/power.mat',t_start = 4.6,
        chemical = chemical_name,
        concentration = concentration,
<<<<<<< HEAD
        extra_time = 9.0,
=======
        extra_time = 6.0,
>>>>>>> public
        dontfit = dontfit,
        run_number = run_number,
        threshold = -50.)
standard_noise_comparison(name)
# tried to fix error on cov more fix more fix more fix more fix more fix
<<<<<<< HEAD
\end{python}
\end{scriptsize}

Check that my longest $T_1$ falls within range I used.
\ntd{just code this in}

Check the consistency of the enhancements with decreasing power.

Comment {\tt search\_delete\_datanode}.

Git commit.

Copy working\_copy into compilation for this project.

Turn $B_0$ off, remove probe + look at sample.

=======
\end{lstlisting}
\end{scriptsize}

\ntd{Here, I should be sure I'm including time for the 3 scans in addition to the ones that are counted.}
>>>>>>> public
\paragraph{Process multiple $T_1$ data}
Copy (or Git once that's ready) files.

Change all parameters at beginning.

Run script

\begin{scriptsize}
<<<<<<< HEAD
\begin{python}[off]
# these change
name = 'dna_cs19_nl_unbound_120418' # name of dataset
chemical = 'dna_cs19' # chemical name
concentration = 0.0 # concentration of spin label (usually 0)
run_number =  120418
number_of_repeats = 30 # number of T1 experiments run
start_exp = 501 # the experiment you started at
=======
\begin{lstlisting}
# these change
name = 'tau_nl_120330'
chemical = 'tau187'
concentration = 0.0
run_number = 120330
number_of_repeats = 7
start_exp = 101
>>>>>>> public
# everything else stays the same
path = DATADIR+'franck_cnsi/nmr/'
dnp_for_rho(path,name,[],expno=[],
    t1expnos = r_[start_exp:start_exp+number_of_repeats],
    integration_width = 150,peak_within = 500,
    show_t1_raw = True,phnum = [4],
    phchannel = [-1],
    chemical = chemical,
    concentration = concentration,
    run_number = run_number,
    h5file='dnp.h5',
    pdfstring = name,
    clear_nodes = False)
t1 = retrieve_T1series('dnp.h5',
    name,
    chemical,
    concentration)
t1.rename('power','expno')
t1.labels('expno',r_[1:1+number_of_repeats])
plot(t1)
lplot(name+'_t1_vs_time.pdf')
<<<<<<< HEAD
\end{python}
\end{scriptsize}
=======
\end{lstlisting}
\end{scriptsize}
\paragraph{put back system}
\subparagraph{Always}
Field off.

Hook up mod coil.

Flip back ESR switch.

Turn off the air + check rate.

Pull out + inspect + measure sample with magnifying glasses.

>>>>>>> public
\subsubsection{ESR}\maxminutes{11}

(Here, I run ESR after the DNP, so I don't have to worry about timing, but still get the double integral)
Put into tune mode, $40\;dB$ first.

Load sample in the teflon sample holder (I want ~77 $mm$ from collet to bottom of sample, need to remeasure) -- weight the top with the collet holder.

Move to near 9.77 and autotune.

Open ESR parameter set similar to this one.
\fn{grocomplex_120202.par}

Turn on field.

Copy experiment.

If not exactly the same type of sample, test run to check some stuff.

\precaution{Only if the experiment says ``uncalibrated'' at the top $\Rightarrow$ ``I'' (interactive spectrometer control) icon, click calibrated, then set parameters to spectrum, then window.}

\paragraph{if new type of sample}
Stop at second peak.

Check that modulation amplitude is $<$ 0.2 x smallest feature.

Set RG with box.

Check that resolution along $x$ is OK.

\paragraph{always}

Run actual scan.

Save ESR.
\fn{groes_120202}

For 8 scans, alarm for about 2.5 min.

Cntrl-S after scan is finished.

Hit Cntrl-A for ssh transfer and process.


\begin{tiny}
\begin{lstlisting}
fl = figlistl()
<<<<<<< HEAD
standard_epr(dir = DATADIR+'cnsi_data',
    files = ['grocomplex_120224','groes_120224','grocomplex_rep_120224'],
=======
standard_epr(dir = DATADIR+'franck_cnsi',
    files = ['tau_likeanna2_120330',
        r'500uM100%label18kdaBeforeHeparin031308'],
>>>>>>> public
    figure_list = fl)
fl.show(thisjobname()+'.pdf')
\end{lstlisting}
\end{tiny}

Put into tune mode before removing sample, and turn down power.

Run background scan.

\subsubsection{wrap up}\maxminutes{18}
<<<<<<< HEAD
Turn off field and air and flip switch.

Pull out, check and measure sample.

Copy this file into compilation for the project.
=======
git commit, then copy working copy of notebook into compilation.
>>>>>>> public

Clear + copy to protocol.

\subparagraph{Run ESR background scan}
<<<<<<< HEAD
\subparagraph{Last experiment only}
Put quartz tube in loops to right, reattach mod coil.

Replace the dewar (being sure to include both inner and outer top collets).

Find dip near 9.88~GHz (if without dewar) or 9.31~GHz (if with dewar), and autotune.

Git commit again and Git push.

=======
Autotune + run ESR background scan.

\subparagraph{Last experiment only}
If done, remove air tube.

Unscrew top collet + carry box w/ dewar around to back.

Remove glass tube and replace dewar.

Insert inner collet.

Screw down top collet holder.

Cap cavity.

Tell software that bridge is on + switch to tune mode

Check for dip near 9.88~GHz (if without dewar) or 9.31~GHz (if with dewar).

Autotune + record Q.

Remove tuning box.

If done, magnet off, ESR off, chiller off, source output off.

Be sure to take coolers and dewar back to lab.
>>>>>>> public

\timeblockend
